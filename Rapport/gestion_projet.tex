\chapter{Gestion du projet}

\section{Présentation du projet}

En lien avec l'équipe ESTER, nous avons reçu, pour tâche, de réaliser un site web permettant de créer des questionnaires médicaux auxquels pourraient répondre des salariés de diverses entreprises. Selon le cahier des charges et tout en respectant le secret professionnel, nous devions faire en sorte que les patients puissent répondre aux questions que les médecins auraient préparé dans des questionnaires, enregistrés dans une base de données. Une fois créé et enregistré, le corps médical devait pouvoir les attribuer en fonction des cas. Par la suite, des résultats devaient être calculé et affiché aux personnels soignants afin qu'ils puissent se rendre compte des chiffres. 

\section{Choix technologies}

\subsection{Technologie côté serveur}


\subsection{Technologie côté client}

JavaScript : Est un langage de programmation développé par Netscape en 1995 sous le nom de LiveScript. Il s'agit d'un langage de script léger, orienté objet, principalement connu comme le langage de script des pages web. \

CSS : « Cascading Style Sheets » ce qui signifie « feuille de style en cascade ». 
Il s'agit d'un langage informatique utilisé pour mettre en forme les fichiers HTML ou XML. \

Bootstrap : Correspond à une collection d'outils, développé depuis 2010, utiles pour la création de sites web. Cette collection contient des codes HTML et CSS ainsi que des extensions JavaScript.

\subsection{Base de données}


\section{Planification et répartition des tâches}

\subsection{Outils utilisés}


\subsection{Diagramme de Gantt}


\subsection{Répartitions des rôles}