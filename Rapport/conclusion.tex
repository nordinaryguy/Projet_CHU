\chapter*{Conclusion}

Ce projet fut pour nous une véritable nouveauté. Encadré par nos chefs de projets avec qui nous avons eu plusieurs réunions et avec qui nous avons souvent échangés, nous avons pu développer un site web assez important. En dépit des difficultés rencontrés, nous avons essayé de mener notre travail jusqu'à son terme. Le projet ESTER fut pour nous l'occasion de découvrir certaines technologies et de les mettre en application, telles que Bootstrap ou surtout JEE. Le cycle de développement que nous avons suivis, nécessitait une bonne organisation. Que ce soit de définir la base de données, de choisir les technologies ou d'implémenter les fonctionnalités dans le temps imparti, il était impossible de passer outre une certaine rigueur dans le planning. La réalisation de ce site nous a également demandés de faire preuve d'une certaine adaptation et de coordination. 
Nous avons pu constater que travailler dans son coin sur un projet d'une telle importance, n'est pas possible, tout comme prendre les tâches les unes après les autres sans compter sur celles à venir. Nous avons partager nos connaissances, nos visions sur le projet et à ce titre, nous avons pu bien plus appréhender le sujet. Par manque de temps, nous n'avons pas faire aboutir totalement le site. Tout de même, nous sommes parvenus à créer une interface utilisateur correspondant aux attentes de notre client. Nous avons développé un générateur de questionnaires que les médecins pourront prendre en main assez aisément et enregistrer leur travaux. Les salariés peuvent répondre aux questionnaires et les utilisateurs sont en mesure d'accéder aux résultats calculés selon les réponses données par l'utilisateur. Enfin, plusieurs autres fonctionnalités ont été ajoutées à ces principales afin que les utilisateurs puissent mieux s'approprier les pages de notre site.