\section{Base de Données}

La base de données est divisée en plusieurs collections.

Pour l'implémentation, nous avons utilisé le driver officiel proposé par MongoDB pour le Java. Nous avons créé une classe mère en Java utilisée par héritage et qui permet de définir des interfaces simplifiant l'utilisation du reste du code.

\subsection{Utilisateur ESTER}

\begin{figure}[H]
    \begin{center}
        \begin{tabularx}{17cm}{|c|X|}
            \hline
            Clé & Valeur  \tabularnewline 
            \hline
            Identifiant & 
            Chaine de caratères, peut être utilisé à la place du mail lors de la connexion \tabularnewline 
            Nom & 
            Chaine de caratères \tabularnewline
            Prénom & 
            Chaine de caratères \tabularnewline
            Première connexion & 
            Boolean \tabularnewline
            Mot de passe & 
            Chaine de caratères chiffrée \tabularnewline
            Mail & 
            Chaine de caratères \tabularnewline
            Statut & 
            Chaine de caratères, représente le statut de l'Utilisateur (Médecin, Administrateur,
            Préventeur, Assistant) \tabularnewline
            \hline
        \end{tabularx}
    \end{center}
    \caption{Tableau de la collection Utilisateur}
\end{figure}

La collection correpond à celle prévue par nos chefs de projet comme
les collections Entreprise et Salarie qui sont des collections proches (L'entreprise ne possède 
pas d'email et le salarie n'a pas de mot de passe mais a la liste des questionnaires répondus ou à répondre) de Utilisateur ESTER donc nous ne nous répéterons pas. 


\subsection{Questionnaire}

\begin{figure}[H]
    \begin{center}
        \begin{tabularx}{17cm}{|c|X|}
            \hline
            Clé & Valeur  \tabularnewline 
            \hline
            Nom & 
            Nom du questionnaire \tabularnewline 
            Identifiant & 
            Version simplifiée du nom \tabularnewline
            Identifiant ESTER & 
            Identifiant de la personne qui a soumis le questionnaire \tabularnewline
            Date de soumission & 
            Date \tabularnewline
            Mail & 
            Chaine de caratères \tabularnewline
            HTML & 
            Chaine de caratères (HTML brut) \tabularnewline
            \hline
        \end{tabularx}
    \end{center}
    \caption{Tableau de la collection Questionnaire}
\end{figure}

\subsection{Réponse}

\begin{figure}[H]
    \begin{center}
        \begin{tabularx}{17cm}{|c|X|}
            \hline
            Clé & Valeur  \tabularnewline 
            \hline
            Identifiant salarie & 
            Identifiant du salarie qui a répondu \tabularnewline
            Identifiant questionnaire & 
            Identifiant du questionnaire répondu \tabularnewline
            Reponses & 
            Tableau associatif qui lie les identifiants des réponses 
            avec leurs réponses \tabularnewline
            \hline
        \end{tabularx}
    \end{center}
    \caption{Tableau de la colection Réponse}
\end{figure}
\subsection{Sécurité}

Nous stockons des mots de passe dans la base de données, nous avons eu besoin de sécuriser les mots de passe pour ne pas les stocker en clair.
Nous avons comparé plusieurs technologies MD5, SHA256, SHA512, PBKDF2, BCrypt et SCrypt. Ce sont toutes des fonctions de hachage mais certaines proposent des bases de salage (PBKDF2, BCrypt, SCrypt) qui permettent de se protéger contre les attaques utilisant des rainbows tables ou par dictionnaire. 
Certaines fonctions de hachage sont difficilement optimisables ce qui permet de rendre plus difficile les attaques par la force brute. De ce fait, cela donne le temps à l'utilisateur de changer son mot de passe.     

Nous avons choisi d'utilisé BCrypt car il propose une implémentation en java et il fait partie des algorithmes les plus sécurisés.