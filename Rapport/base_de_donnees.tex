\section{Base de Données}

La basse de données est divisé en collections 

Pour l'implémentation nous avons utilisé le driver officiel proposer par MongoDB pour le Java. Nous avons crée un classe en Java qui nous héritons et que permet des interfaces pour simplifier l'utilisation dans le reste du code.

Nous avons divisé notre base de données en plusieurs collections. 

\subsection{Utilisateur ESTER}

\begin{figure}[H]
    \begin{center}
        \begin{tabularx}{17cm}{|c|X|}
            \hline
            Clée & Valeur  \tabularnewline 
            \hline
            Identifiant & 
            Chaine de caratères peut être utilisé à la place du mail lors de la connexion \tabularnewline 
            Nom & 
            Chaine de caratères \tabularnewline
            Prénom & 
            Chaine de caratères \tabularnewline
            Première connexion & 
            Boolean \tabularnewline
            Mot de passe & 
            Chaine de caratères chiffré \tabularnewline
            Mail & 
            Chaine de caratères \tabularnewline
            Statut & 
            Chaine de caratères représente le status du salarié (Medecin, Administrateur,
            Preventeur, Assistant) \tabularnewline
            \hline
        \end{tabularx}
    \end{center}
    \caption{Tableau de la collection Utilisateur ESTER}
\end{figure}

La collection correpond à celle prévue par nos chefs de projet comme
les collections Entreprise et Salarie qui sont des collections proches (L'entreprise ne possède 
pas d'email et le salarie n'a pas de mots de passe mais a la liste des questionnaires réponds ou a répondre
) de Utilisateur ESTER donc nous nous répéterons pas. 

\subsection{Questionnaire}

\begin{figure}[H]
    \begin{center}
        \begin{tabularx}{17cm}{|c|X|}
            \hline
            Clée & Valeur  \tabularnewline 
            \hline
            Nom & 
            Mom du questionnaire \tabularnewline 
            Identifiant & 
            Version simplifier du nom \tabularnewline
            Identifiant ESTER & 
            Identifiant de la personne qui a soumis le questionnaire \tabularnewline
            Date de soumission & 
            Date \tabularnewline
            Mail & 
            Chaine de caratères \tabularnewline
            HTML & 
            Chaine de caratères (HTML brut) \tabularnewline
            \hline
        \end{tabularx}
    \end{center}
    \caption{Tableau de la collection questionnaire}
\end{figure}

\subsection{Réponse}

\begin{figure}[H]
    \begin{center}
        \begin{tabularx}{17cm}{|c|X|}
            \hline
            Clée & Valeur  \tabularnewline 
            \hline
            Identifiant salarie & 
            Identifiant du salarie qui a répondu \tabularnewline
            Identifiant questionnaire & 
            Identifiant du questionnaire répondu \tabularnewline
            Reponses & 
            Tableau associatif qui lie les identifiants des reponses 
            avec leurs reponses \tabularnewline
            \hline
        \end{tabularx}
    \end{center}
    \caption{Tableau de la colection réponse}
\end{figure}

\subsection{Sécurité}

Car nous stockons des mots de passe dans la base de données, nous avons eu besoin de sécuriser les mots de passe pour ne pas les stocker en clair.

Nous avons comparé plusieurs technologies MD5, SHA256, SHA512, PBKDF2, BCrypt et SCrypt. Ce sont tous des fonctions de hachage mais certaine propose de base le salage (PBKDF2, BCrypt, SCrypt) qui permet de ce protéger contre les attaques utilisant des rainbow tables ou par dictionnaire. 

Certaine fonction de hachage sont difficilement optimisable ce qui permet de rendre plus difficile les attaques par brute force ce qui permettra de donné le temps a l'utilisateur de changer sont mots de passe.     

Nous avons choisi d'utilisé BCrypt car il propose un implémentation en java et il fais parti des plus sécuriser de algorithme.
